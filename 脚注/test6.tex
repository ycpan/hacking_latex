\documentclass[twoside]{article} % Note: uses twoside option
\usepackage[a4paper, marginparwidth=75pt, total={10cm, 10cm}]{geometry} % To create a small page
% 定义一个控制边注显示的变量
%\newif\ifshowmarginnotes
% 默认情况下显示边注
\showmarginnotestrue
% 如果要隐藏边注,使用
\showmarginnotesfalse
\usepackage{hyperref} % To use the \url command (in the footnote)
\usepackage{marginnote}
\begin{document}
\section{Lorem Ipsum}
\footnote{Source text: Wikipedia (\url{https://en.wikipedia.org/wiki/Lorem_ipsum})}But I must explain to you how all this mistaken idea of reprobating pleasure and extolling pain arose. To do so, I will give you a complete account of the system, and expound the actual teachings of the great explorer of the truth, the master-builder of human happiness. 
% 根据条件显示边注
\ifshowmarginnotes
\marginpar[c]{Note 1: text for right-hand side of pages, it is set justified.}
\else
  % 如果不显示边注,可以在这里放置其他内容,或者留空
\fi

No one rejects, dislikes or avoids pleasure itself, because it is pleasure, but because those who do not know how to pursue pleasure rationally encounter consequences that are extremely painful. Nor again is there anyone who loves or pursues or desires to obtain pain of itself, because it is pain, but occasionally circumstances occur in which toil and pain can procure him some great pleasure.  
% 根据条件显示边注
\ifshowmarginnotes
\marginpar{\raggedright Note 2: text for right-hand side of pages, it is not justified, but uses \texttt{\string\raggedright}.} 
\else
  % 如果不显示边注,可以在这里放置其他内容,或者留空
\fi
To take a trivial example, which of us ever undertakes laborious physical exercise, except to obtain some advantage from it? But who has any right to find fault with a man who chooses to enjoy a pleasure that has no annoying consequences, or one who avoids a pain that produces no resultant pleasure? 
\end{document}

