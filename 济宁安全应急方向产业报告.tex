
\documentclass[UTF8,twocolumn]{ctexart}
\usepackage{hyperref}%用于识别autoref
\usepackage[margin=1in]{geometry} % 设置所有边距为1英寸
\usepackage{indentfirst} %首行缩进
\usepackage{longtable}% 导入跨页表格所需宏包
\usepackage{booktabs}%引入跨页三线
\usepackage{multirow}
\usepackage{caption} %设置图或者表的标题
\usepackage{multicol}
\usepackage{lipsum} %用来生成占位文本
\usepackage{pgfplots}%用来画折现图
\pgfplotsset{width=10cm,compat=1.9}
\usepackage{xcolor} % 用于定义颜色 % 定义颜色
\definecolor{headerfootercolor}{RGB}{187,51,24} % 红色
\usepackage{fancyhdr} % 设置页眉和页脚样式
\pagestyle{fancy}
\fancyhf{} % 清除现有的页眉和页脚设置
\fancyhead[R]{\includegraphics[height=0.6cm]{./pic/logo.png}} % 左侧页眉插入logo
\fancyfoot[C]{
  \begin{minipage}{\textwidth}
    \centering
    \footnotesize\textcolor{gray}{合享汇智集团}\\
    \footnotesize\textcolor{gray}{北京市海淀区北四环西路52号方正国际大厦804}\\
    \footnotesize\textcolor{gray}{电话:010-82092128  邮编:100080  网址:incoshare.com}
  \end{minipage}
  }
\fancyfoot[R]{\textcolor{black}\thepage} % 中间页脚,显示页码
\renewcommand{\headrulewidth}{0.5pt}
\renewcommand{\headrule}{\color{headerfootercolor}\hrule width\headwidth height\headrulewidth \vskip1pt \hrule width\headwidth height\headrulewidth}
\renewcommand{\footrulewidth}{0.5pt}
\renewcommand{\footrule}{\color{headerfootercolor}\rule{\textwidth}{\footrulewidth}}


\begin{document}
\title{杭州人工智能产业报告}
\author{产业AI助手}
\twocolumn[
\begin{@twocolumnfalse}%全局双栏中部分单栏,非常重要
\maketitle


济宁安全应急产业发展势头强劲,企业数量持续增长。截至目前,济宁市安全应急方向的企业数量达到1837家,其中上市企业有362家,小巨人企业478家,科创企业997家。近年来,济宁安全应急产业呈现高速增长态势,近五年来企业数量增长情况如下:2017年增长7家,2018年增长251家,2019年增长727家,2020年增长744家,2021年增长849家。这充分展示了济宁安全应急产业的活力与发展潜力。
\section{安全应急方向济宁企业以及分属类别}
表\ref{表:安全应急方向济宁企业以及分属类别}显示了杭州市人工智能企业按照不同类型的数量统计,从表\ref{tab:安全应急方向济宁企业以及分属类别}可以看出:
\begin{table}[htbp]
  \centering
    \begin{tabular}{c | c}
      \toprule
      企业类型 & 数量 + \\
          安全应急方向济宁的企业数量 & 1837\\
      \midrule
      安全应急方向济宁上市的企业数量 & 362\\
      \midrule
      安全应急方向济宁小巨人的企业数量 & 478\\
      \midrule
      安全应急方向济宁科创的企业数量 & 997\\
      \bottomrule

  \end{tabular}
  \caption{title}
  \label{表:安全应急方向济宁企业以及分属类别}
\end{table}
    根据数据显示,济宁在安全应急领域的企业数量达到了1837家,这表明济宁市政府对安全应急产业的高度重视和大力扶持,同时说明济宁在安全应急产业方面具有较强的产业基础和发展潜力。其中,上市企业数量为362家,这些企业不仅在资本市场上崭露头角,而且在技术研发、市场开拓、行业引领等方面具备较强的实力,对整个产业的发展起到了良好的示范和带动作用。此外,小巨人企业数量为478家,这些企业具有较高的成长性、创新性和市场竞争力,是济宁安全应急产业的中坚力量。而科技创新企业数量为997家,这些企业以技术创新为核心,是推动产业转型升级的重要力量。总体来看,济宁安全应急产业的企业数量分布合理,具有较好的发展均衡性,有望打造成为我国安全应急产业的重要基地。
\section{济宁安全应急方向近五年增长情况}
图\ref{fig:济宁安全应急方向近五年增长情况}是济宁安全应急方向企业近五年的增长情况,从中我们可以进行以下几点分析:
\begin{figure}[htbp]
  \centering
  \begin{tikzpicture}[scale=0.7]
    \begin{axis}[
      title={济宁安全应急方向近五年增长情况},
      xlabel={年份},
      ylabel={企业数量},
      xmin=2017, xmax=2021
      ymin=0, ymax=933,
      xtick={2017,2018,2019,2020,2021},
      ytick={7,251,727,744,849},
      legend pos=north west
    ]
    \addplot[
      color=blue,
      mark=square,
      ]
      coordinates {
        %(2017,10)(2018,25)(2019,33)(2020,60)(2021,120)(2022,160)
        (2017, 7)(2018, 251)(2019, 727)(2020, 744)(2021, 849)
      };
    \legend{济宁安全应急方向企业数量}
    \end{axis}
  \end{tikzpicture}
  \caption{济宁安全应急方向近五年增长情况}
  \label{fig:济宁安全应急方向近五年增长情况}
\end{figure}
    济宁安全应急方向企业在过去五年中实现了显著的增长,体现了我国在安全应急领域的发展态势和重视程度。2017年,该产业增加值仅为7,然而在2018年迎来了爆发式增长,达到了251,之后每年都保持了较高的增长速度,2019年增至727,2020年达到744,2021年更是高达849。这一系列数据充分展示了济宁安全应急产业的发展潜力和活力,同时也表明了我国在应对突发公共安全事件方面的能力提升。安全应急产业作为一项涉及公共安全、人民生命财产安全的重要产业,其快速发展对于提升我国应急管理水平、构建和谐社会具有重要意义。
\section{结语}
济宁安全应急产业发展迅速,企业数量稳步增长。截至目前,共有1837家企业从事安全应急方向的工作,其中362家上市公司,478家被认定为小巨人,997家属于科创企业。近年来,济宁安全应急产业呈现显著增长态势,近五年增长率分别为7、251、727、744和849,显示出该产业强大的发展潜力和市场活力。

\end{document}
